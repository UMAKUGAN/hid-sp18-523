\section{Google Dremel}
\index{Google Dremel}


With vast amount of publicly available data over the internet/cloud, 
there was a need of technological system/framework that is deployed on 
cloud which can execute on demand queries in faster and scalable way 
for read only multi level nested data. Along with that a system that 
uses structured query language, which is widely adapted and extensively 
used by the developers for writing queries to avoid the learning curve of 
new language. To fill this gap Google came up with Dremel. It is a 
interactive ad hoc query system that lets the user query the large 
dataset providing them results with much faster speed compared to 
traditional technologies~\cite{hid-sp18-523-www-dremel}. ``By combining 
multi-level execution trees and columnar data layout, it is capable of 
running aggregation queries over trillion-row tables in 
seconds''~\cite{hid-sp18-523-www-dremel}. ``Dremel is capable of scaling 
up to thousands of CPUs and petabytes of data''~\cite{hid-sp18-523-www-dremel}.
MapReduce framework and technologies thar are built over it such as Pig, Hive 
etc has latency issue between running the job and getting output. 
Dremel on the other hand took a different approach, it uses execution engine
based on on aggregating trees algorithm that provides output almost realtime 
for queries.
