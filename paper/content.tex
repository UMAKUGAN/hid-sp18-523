% status: 0
% chapter: TBD

\title{Domo}

\author{Ritesh Tandon}
\affiliation{%
 \institution{Indiana University}
 \city{Bloomington} 
 \state{IN} 
 \postcode{47408}
 \country{USA}}
\email{ritandon@iu.edu}


% The default list of authors is too long for headers}
\renewcommand{\shortauthors}{R. Tandon}


\begin{abstract}
Domo is cloud based data integration platform that enables employees 
to engage with data that is located anywhere in real time. 
It provides flexibility to outside partners,third party vendors to 
integrate and collaborate with data. With more than 400 data connectors,
data can be accessed easily directly from public or private cloud,
on-premise or proprietary systems.
\end{abstract}

\keywords{hid523,Domo,API,OAuth2,SQL,Json}

\maketitle


\section{Introduction}

Data is the heart of information of any business. 
Though it might seems easy,but it is very trivial 
to find the relevant data that is required by different 
people working in different departments of large organization. 
More so,the bigger challenge is to derive the insights 
from the data when it is located.
Domo transforms the way orgaization's employee access,use,
analyze and share data. Domo gives the power to users to 
make decision in real time. Domo can be thought of as cloud 
based data operating system that has the ability to handle 
and process data regardless of its type and location. 
Domo brings different data sources spread across different 
locations at one central location so that it can be easily 
accessible for use. Domo also let the user share and 
colloborate different data sources,visualization and 
reports. It also has realiable data management feature 
that provides high level of security,speed and scalability. 
Domo makes data avialble on any device of any size thus 
making it truly mobile.

\section{Domo- Inbuilt Solutions}

Domo has custom inbuilt dashboard and visualization solution 
for different roles (such as BI,CEO,Finance,IT,Marketing,
Operations,Sales,Services etc)  within oroganization and 
for different industries (such as Education,Healthcare,
Manufacturing,Hospitatlity,Retail,Transportation etc)


\section{Data Connectors}
Data Connectors is the heart of Domo. Through Domo we may 
connect to many different types of data sources. Domo supports 
connecting to many types of data sources such as  Cloud App,
File,Database,On Premise,Api.

Cloud App Connectors; Domo has  more than 400 cloud app 
connectors including all famous ones such as amazon s3,
AWS,Adobe analytics,Google Analytics,Facebook,fitbit,
instagram,salesforce etc. 

File Connectors; Through Domo we can also connect to data 
that is stored in files such as excel and/or csv files.

Database; Domo has connectors for connecting relational,
non relational,SQL and NO-SQL databases such as Oracle,
MS SQL,MySQL,Hadoop etc.

On Premise; Other than cloud based connectors Domo can also 
connect to on premise databases/files etc as long as security 
protocols are opened securely for connection.

Api; Domo has Dev Studio tool for creating custom apps. 
It is best suited for developers having web development 
experince (java script,css,html). Domo App CLI is the 
main tool that is used to create,edit and publish app designs 
to the Domo instance.


\section{Data Flows and Transforms}

Cleaning data is herculean task when dealing with data that 
is dirty that needs to be cleaned before reporting. Domo has 
Magic ETL tool that makes data cleaning job looks easy. 
It helps join,transform and tidy up data with drag and 
drop ease of use~\cite{hid-sp18-523-Dev} 

Domo also has SQL data flow that let the developer select 
data set,perform transformation operation through SQL query 
and generate tidy and processed output dataset. Domo also 
give option to run the dataflow whenever dataset is updated 
thus making sure that final visualization and report is 
always based on latest clean data in almost real time.


\section{Visualization}

Domo has many inbuilt visualization template that helps the 
user present the user story in refined visual format.
These predefined template are called Cards in Domo. Horizontal bar,
Vertical bar,Line,Area,Data Science,Pie and Funnel are 
few popular visualization categories. These individual categories 
contain many use ful templates for e.g Data Science category 
has visualization template for scatter plot,box plot,predictive 
modeling,outliers etc to visually represent relevant data. 
Donut,Pie,Treemap,Funnel,Folded funnel are few of the
popular visualization template under this category.



\section{How It Works}
Create data connector as needed (file,cloud,on premise,Api etc)

Select the connector and create required dat set by 
selecting specifying table,views or by custom sql query.

Select the dataset and chose visualization card under 
respective category (Bar,Pie,Funnel,Scatter,
Predective modeling etc)

Drag and drop the fields/attributes that are needed 
in visualization/report 

Apply inbuilt aggregate function on fields as needed 

Save the card. Move to dashborad if needed.

Give access and share your visulization card with concerned users.


\section{Dev Studio}

Integrated development environment that provides developers 
with web developemt experience to create custom apps that 
can be deployed in Domo instance easily. Development environment 
consists of following three main componenets

Domo App CLI is Used to create,edit and publish custom app 
on Domo environment

App Design is Custom built template that can be connected to 
different datasets and visualize data (This can be used when 
there is need of custom visulization requirement for which 
standard template is not available)

App Manifest is Configuration file that defines properties 
of custom app


\section{Installation}

Install node.js through download

Install CLI using npm install -g ryuu command on unix/linux based platform. 
Make sure that firewall is not blocking npm registry by pining 
www.npmjs.com through terminal


\section{Creating simple Domo App}
On the CLI Type command  domo init on terminal.This will initiate basic 
design template

We will be asked to enter design name and starter type App. 
Enter myfirstdomoapp as design name and HelloWorld as starter type. 
This will create directory and all the necesary files that we need for 
building simple app

\begin{verbatim}
  ``Following project structure is created -
      app.cs
      app.js
      domo.js
      index.html
      manifest.json``
\end{verbatim}
~\cite{hid-sp18-523-Dev}

Skip the data source connection part as we are building simple
custom app that can be deployed on Domo instance

From the CLI run domo dev command. This will open browser and will 
render myfirstdomoapp

Make the styling changes in app.css and logic changes in app.js

\section{API Authentication}

Security of data that is transmitted over wire is of highest
importance to any organization. Any public API is expected to 
validate and authenticate only those clients that have access.
Domo API uses OAuth2.0 for authenticating and authorizing clients.
Security is managed through access tokens. Only authenticated
and authorized users can get tokens. For accessing Domo API through
OAuth security client program must obtain ClientId and client Secret.
Once authenticated users can access API functionality through access
token.
To create Client,Login to Domo instance and click on create new 
client link under user avatar icon. Specify application name and
description. Choose one or more from Audit,Data,Dashboard and User
application scope as applicable. We have to be careful while choosing
application scope; if application scope is only for accessing data we
should only select Data scope else developes will get access to user,
audit related informaion as well. Once Client Id and Client Secret is
obtained,next step to obtain access token. We can make following request
to obtain access token using Id and secret~\cite{hid-sp18-523-Authticate}
\begin{verbatim}
``$curl -v -u {CLIENT_ID}:{CLIENT_SECRET} 
'https://api.domo.com/oauth/token?
	grant_type=client_credentials&scope={SCOPE}'``
\end{verbatim}
~\cite{hid-sp18-523-Authticate}

Once we request the token using above command we will get the JSON response.
Body of JSON response will contain multiple key value pairs. The most
important among those are access token and expires in key.Obtained access 
token must be passed in header of any future request.

For e.g Use below command if we wish to call Domo API that gives us list of 
datasets that we have created after replacing the access token that we have
obtained
\begin{verbatim}
``$curl -v -H Authorization:'bearer {access-token}' 
'https://api.domo.com/v1/datasets'``
\end{verbatim}
~\cite{hid-sp18-523-Authticate}

We can build our custom app using Domo API as exlained above.


\section{Data API}
Base url (end point) of the data API can be accessed through following
command
\begin{verbatim}
``GET /data/v1/:alias?:queryOperators``
\end{verbatim}
~\cite{hid-sp18-523-DataApi}
Alias is the name of the dataset that we have defined in our manifest
file.We can define and run our custom query using queryOperators. We can
pass aggregate functions such as count,sum,min,max,avg,filter,groupby,
orderby etc. We can control the format of the returned data of API by
setting the request accept header of XMLHttpRequest object to following 
formats~\cite{hid-sp18-523-DataApi} 
\begin{verbatim}
``array-of-objects
  csv
  excel
  json``
\end{verbatim}
~\cite{hid-sp18-523-DataApi}

We can also specify return format in domo.get method.


\section{Multi User API}
Domo offer following end point for accessing information is all
Domo instance users
\begin{verbatim}
``GET /domo/users/v1?includeDetails={true|false}
	&limit={int}&offset={int}``
\end{verbatim}
~\cite{hid-sp18-523-User}
While calling API we can control user details returned by the API,limit
the number of records we want the API to return and specify offset to get
the list of users starting from given offset. 
~\cite{hid-sp18-523-User} 

\section{Single User API}
Domo offer following end point for accessing information of single user
\begin{verbatim}
``GET /domo/users/v1/:userId?
	includeDetails={true|false}``
\end{verbatim}
~\cite{hid-sp18-523-User}
We can pass the user id of whom we needs details or pass the current user
accessing through environment variable.While calling API we can control 
user details returned by the API~\cite{hid-sp18-523-User} 


\section{Sharing custom app using Domo}
We can share our custom app/visualization card/report etc with other users
by logging in to Domo CLI through domo login command and then publish the 
custom built app on domo instance using domo publish command. 


\section{Conclusion}
Domo is used as cloud based tool for real time data visualization and 
reporting. Through Dev Studio and public API,Domo lets the developer
extends the capability of customizing visualization and build reporting
template that may be used for building custom app.Domo business cloud
platform offers high availability,performance and scalability for the
applications that deployed on Domo instance.


\begin{acks}

Author would like to thank Dr Gregor Von Laszewski for his suggestions 
and guidance on the content that is presented in paper.
\end{acks}

\bibliographystyle{ACM-Reference-Format}
\bibliography{report} 

