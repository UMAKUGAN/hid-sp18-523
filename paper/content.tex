% status: 0
% chapter: TBD

\title{Domo}

\author{Ritesh Tandon}
\affiliation{%
 \institution{Indiana University}
 \city{Bloomington} 
 \state{IN} 
 \postcode{47408}
 \country{USA}}
\email{ritandon@iu.edu}


% The default list of authors is too long for headers}
\renewcommand{\shortauthors}{R. Tandon}


\begin{abstract}

Domo is cloud based data integration platform that enables employees 
to engage with data that is located anywhere in real time. 
It provides flexibility to outside partners, third party vendors to 
integrate and collaborate with data. With more than 400 data connectors, 
data can be accessed easily directly from public or private cloud, 
on-premise or proprietary systems.

\end{abstract}

\keywords{hid523, Domo}

\maketitle


\section{Introduction}

Data is the heart of information of any business. 
Though it might seems easy, but it is very trivial 
to find the relevant data that is required by different 
people working in different departments of large organization. 
More so, the bigger challenge is to derive the insights 
from the data when it is located.
Domo transforms the way orgaization's employee access, use, 
analyze and share data. Domo gives the power to users to 
make decision in real time. Domo can be thought of as cloud 
based data operating system that has the ability to handle 
and process data regardless of its type and location. 
Domo brings different data sources spread across different 
locations at one central location so that it can be easily 
accessible for use. Domo also let the user share and 
colloborate different data sources , visualization and 
reports. It also has realiable data management feature 
that provides high level of security , speed and scalability. 
Domo makes data avialble on any device of any size thus 
making it truly mobile.

\section{Domo- Inbuilt Solutions}

Domo has custom inbuilt dashboard and visualization solution 
for different roles ( such as BI, CEO, Finance, IT, Marketing, 
Operations, Sales, Services etc )  within oroganization and 
for different industries ( such as Education, Healthcare, 
Manufacturing, Hospitatlity, Retail, Transportation etc )


\section{Domo - Data Connectors}
\begin{itemize}
\item Data Connectors is the heart of Domo. Through Domo we may 
connect to many different types of data sources. Domo supports 
connecting to many types of data sources such as  Cloud App, 
File, Database, On Premise, Api.

\item Cloud App Connectors - Domo has  more than 400 cloud app 
connectors including all famous ones such as amazon s3, 
AWS, Adobe analytics, Google Analytics, Facebook, fitbit, 
instagram, salesforce etc. 

\item File Connectors - Through Domo we can alsoconnect to data 
that is stored in files such as excel and/or csv files.

\item Database - Domo has connectors for conneting relational ,
non relational, SQL and NO-SQL databases such as Oracle, 
MS SQL, MySQL, Hadoop etc.

\item On Premise - Other than cloud based connectors Domo can also 
connect to on premise databases/files etc as long as security 
protocols are opened securely for connection.

\item Api - Domo has Dev Studio tool for creating custom apps. 
It is best suited for developers having web development 
experince ( java script, css , html ). Domo App CLI is the 
main tool that is used to create, edit and publish app designs 
to the Domo instance.
\end{itemize}

\section{Domo - Data Flows and Transforms}

Cleaning data is herculean task when dealing with data that 
is dirty that needs to be cleaned before reporting. Domo has 
Magic ETL tool that makes data cleaning job looks easy. 
It helps join, transform and tidy up data with drag and 
drop ease of use. 

Domo also has SQL data flow that let the developer select 
data set, perform transformation operation through SQL query 
and generate tidy and processed output dataset. Domo also 
give option to run the dataflow whenever dataset is updated 
thus making sure that final visualization and report is 
always based on latest clean data in almost real time.


\section{Domo - Visualization}

Domo has many inbuilt visualization template that helps the 
user present the user story in refined visual format.
These predefined template are called Cards in Domo. Horizontal bar, 
Vertical bar, Line, Area , Data Science , Pie and Funnel are 
few popular visualization categories. These individual categories 
contain many use ful templates - for e.g - Data Science category 
has visualization template for scatter plot, box plot, predictive 
modeling , outliers etc to visually represent relevant data. 
Donut, Pie, Treemap, Funnel, Folded funnel are few of the
popular visualization template under this category.



\section{How It Works}
\begin{itemize}
\item Create data connector as needed 
( file , cloud, on premise , Api etc )
\item Select the connector and create required dat set by 
selecting specifying table, views or by custom sql query.
\item Select the dataset and chose visualization card under 
respective category ( Bar, Pie, Funnel , Scatter, 
Predective modeling etc )
\item Drag and drop the fields/attributes that are needed 
in visualization/report 
\item Apply inbuilt aggregate function on fields as needed 
\item Save the card. Move to dashborad if needed.
\item Give access and share your visulization card with concerned users.

\end{itemize}

\section{Domo - Dev Studio}

Integrated development environment that provides developers 
with web developemt experience to create custom apps that 
can be deployed in Domo instance easily. Development environment 
consists of following three main componenets-
\begin{itemize}
\item Domo App CLI - Used to create , edit and publisg custom app 
on Domo environment
\item App Design - Custom built template that can be connected to 
different datasets and visualize data ( This can be used when 
there is need of custom visulization requirement for which 
standard template is not available )
\item App Manifest - Configuration file that defines properties 
of custom app
\end{itemize}

\section{Installation}
\begin{itemize}
\item Install node.js through download
\item Install CLI using ``npm install -g ryuu`` on unix/linux based platform. 
   Make sure that firewall is not blocking npm registry by pining
   www.npmjs.com through terminal
\end{itemize}


\section{Creating simple Domo App}
\begin{itemize}
\item On the CLI Type command  ``domo init`` on terminal.
   This will initiate basic design template
\item We will be asked to enter design name and starter type App. 
   Enter ``myfirstdomoapp`` as design name and ``HelloWorld`` as 
   starter type. This will create directory and all the necesary
   files that we need for building simple app

      ``Following project structure is created -
        app.cs
        app.js
        domo.js
        index.html
        manifest.json``~\cite{hid-sp18-523-Dev}

\item Skip the data source connection part as we are building simple
   custom app that can be deployed on Domo instance
\item From the CLI run ``domo dev``. This will open browser and will 
   render myfirstdomoapp.
\item Make the styling changes in app.css and logic changes in app.js

\end{itemize}


\section{Sharing custom app using Domo}
\begin{itemize}
\item Login to Domo with CLI by typing ``domo login``
\item publish the custom built app on domo instance
   using ``domo publish``
\end{itemize}


\section{Conclusion}
Domo is used as cloud based tool for real time data visualization and 
reporting. Through Dev Studio and public API, Domo lets the developer
extends the capability of customizing visualization and build reporting
template that may be used for building custom app.Domo business cloud
platform offers high availability, performance and scalability for the
applications that deployed on Domo instance.


\begin{acks}

Author would like to thank Dr Gregor Von Laszewski 
for his suggestions and guidance on that content to be presented
in paper.
\end{acks}

\bibliographystyle{ACM-Reference-Format}
\bibliography{report} 

